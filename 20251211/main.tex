%%=======================================================================
% !Mode:: "TeX:UTF-8"
% !TEX program  = XeLaTeX
%%=======================================================================
% 模板名称:thubeamer
% 模板版本:V1.2
% 模板作者:杨敬轩(Jingxuan Yang)
% 联系作者:yangjx20@mails.tsinghua.edu.cn & yanglatex2e@gmail.com
% 模板适用:清华大学风格 Beamer 模板
% 模板编译:手动编译方法参看 README.md 或 thubeamer.pdf
%          编译 beamer 之前必须编译说明文档:make doc 或双击 makedoc.bat
%          编译说明文档同时分离出四个样式文件 *thubeamer.sty
%          GNU make 工具:make beamer
%          Windows 批处理脚本:双击 makebeamer.bat 自动编译 beamer
%          更多编译细节详见说明文档:thubeamer.pdf
% 更新时间:2023/11/27
% 模板帮助:请**务必务必务必**阅读 thubeamer.pdf 说明文档,文档查看方法:
%          下载模板文件夹里就有,如果是从 CTAN 上安装更新本模板,则通过
%          cmd 命令行:texdoc thubeamer 查看文档
%          推荐前往模板的 GitHub 仓库获取最新文件,地址:
%          https://github.com/YangLaTeX/thubeamer
%%=======================================================================

% 设置文档类别为 <beamer>
% \documentclass[aspectratio=169]{beamer} % 设置长宽比为 16:9
\documentclass{ctexbeamer}

% 使用 <thubeamer> 主题
% 模板选项如下
% (a.1) smoothbars: 页面顶端单行显示目录,默认选项
\usetheme[sectiontoc]{thubeamer}


% (a.2) sidebar: 页面左侧分栏显示目录
% \usetheme[sidebar]{thubeamer}

% (b) sectiontoc: 在每节(section)前显示目录,并高亮显示当前节,默认不显示

% (c) subsectiontoc: 在每小节(subsection)前显示目录,并高亮显示当前节和当前小节,默认不显示

% (d) en: 仅使用英文来制作 beamer,应用此选项后,汉字将全部无法编译

% 图片存放路径

\graphicspath{{figures/}}

\include{macros}

% 封面信息,方括号内容是显示在左侧边栏的内容(当选择 sidebar 主题时有效)
\title[篮球高光检测]{篮球高光检测项目\\[2mm] 汇报}
\author[唐梓楠]{牛衍昌\\鲁继元\\唐梓楠\\[5mm] 导师:黄必清教授}
\institute[清华大学]{\small 清华大学}
\date{\small \vskip -10pt \today}

% 开始写文章
\begin{document}

% 标题页
\begin{frame}
  \maketitle
\end{frame}

% 目录页
\section*{目录}
\frame{
  \frametitle{\secname}
  \tableofcontents[hideallsubsections]
}

\section{Yolo12x 训练}

\begin{frame}{SportsMOT 数据集}

  \begin{block}{ICCV 2023}
    \textbf{SportsMOT: A Large Multi-Object Tracking Dataset in Multiple Sports Scenes}~\cite{Cui_2023_ICCV}
  \end{block}

  专业比赛运动场景的多目标追踪数据集,包括\textbf{篮球、排球、足球}。

  包含 \textbf{240 个三个类别的片段},均从 \textbf{YouTube} 上的\textbf{奥运会、NCAA 和 NBA} 中收集。仅下载\textbf{分辨率 720P}、\textbf{帧率 25 FPS} 且为\textbf{官方录制}的搜索结果。所有视频被手动剪辑为\textbf{平均 485 帧}的片段,片段内无镜头切换。

  足球比赛提供户外场景,其余为室内场景。比赛场地的视角各不相同,如 NBA 观众密集的侧面视角、排球发球区视角、足球空中俯视视角等。
\end{frame}

\begin{frame}{SportsMOT 数据集 Statstics}
  \begin{table}[h]
    \centering
    \resizebox{\linewidth}{!}{
    \begin{tabular}{lccccc}
      \toprule
      Category (avg.) & \#frames & \#tracks & track gap len & track length & \#bboxes per frame \\
      \midrule
      篮球 & 845.4 & 10 & 68.7 & 767.9 & 9.1 \\
      排球 & 360.4 & 12 & 38.2 & 335.9 & 11.2 \\
      足球 & 673.9 & 20.5 & 116.1 & 422.1 & 12.8 \\
      \bottomrule
    \end{tabular}
    }
  \end{table}

  \begin{columns}
    \begin{column}{0.5\textwidth}
      \begin{figure}[t]
        \centering
        \includegraphics[width=0.8\linewidth]{speed.png}
        \caption{速度分布(高斯概率密度函数)}
      \end{figure}
    \end{column}
    \begin{column}{0.5\textwidth}
      \begin{itemize}
        \item \textbf{track}: 每个视频的轨迹数量
        \item \textbf{tracklen}: 每个视频的平均长度/帧数
        \item \textbf{speed}: 视频中运动员的平均速度
      \end{itemize}
    \end{column}
  \end{columns}
\end{frame}

\begin{frame}{SportsMOT 数据集 Cases}
  \begin{columns}
    \begin{column}{0.33\textwidth}
      \begin{figure}[htbp]
        \centering
        \centering
        \includegraphics[width=\linewidth]{Basketball.png}
        \begin{center} \scriptsize 篮球(NBA) \end{center}
      \end{figure}
    \end{column}
    \begin{column}{0.33\textwidth}
      \begin{figure}[htbp]
        \centering
        \includegraphics[width=\linewidth]{Volleyball.png}
        \begin{center} \scriptsize 排球(2012 伦敦奥运会) \end{center}
      \end{figure}
    \end{column}
    \begin{column}{0.33\textwidth}
      \begin{figure}[htbp]
        \centering
        \includegraphics[width=\linewidth]{Football.png}
        \begin{center} \scriptsize 足球(英格兰足总杯) \end{center}
      \end{figure}
    \end{column}
  \end{columns}
\end{frame}

\begin{frame}{Yolo12x 训练情况}
  \begin{columns}
    \begin{column}{0.5\textwidth}
      \begin{itemize}
        \item 先在整个数据集上训练
        \item 考虑到足球、排球与篮球场景差异较大,后尝试单独在篮球上训练
        \item 效果都变差
      \end{itemize}
      \textbf{可能的原因}:
      \begin{itemize}
        \item Yolo12x 本身足够强
        \item SportsMOT 为 720p,测试数据为 1080p
        \item SportsMOT 为专业球场,测试数据为野球场
      \end{itemize}
    \end{column}
    \begin{column}{0.5\textwidth}
      \begin{figure}
        \centering
        \includegraphics[width=\linewidth]{train.png}
        \caption{Yolo12x 训练前后对比}
      \end{figure}
    \end{column}
  \end{columns}
\end{frame}

\section{畸变校正}

\section{连续性优化}

\begin{frame}[allowframebreaks]{参考文献}
  \bibliographystyle{thubeamer}
  \bibliography{reference}
\end{frame}

\begin{frame}
  \begin{center}
    {\Huge\calligra Thanks for your attention!}
  \end{center}
\end{frame}

% 结束文档撰写
\end{document}
