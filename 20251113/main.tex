%%=======================================================================
% !Mode:: "TeX:UTF-8"
% !TEX program  = XeLaTeX
%%=======================================================================
% 模板名称:thubeamer
% 模板版本:V1.2
% 模板作者:杨敬轩(Jingxuan Yang)
% 联系作者:yangjx20@mails.tsinghua.edu.cn & yanglatex2e@gmail.com
% 模板适用:清华大学风格 Beamer 模板
% 模板编译:手动编译方法参看 README.md 或 thubeamer.pdf
%          编译 beamer 之前必须编译说明文档:make doc 或双击 makedoc.bat
%          编译说明文档同时分离出四个样式文件 *thubeamer.sty
%          GNU make 工具:make beamer
%          Windows 批处理脚本:双击 makebeamer.bat 自动编译 beamer
%          更多编译细节详见说明文档:thubeamer.pdf
% 更新时间:2023/11/27
% 模板帮助:请**务必务必务必**阅读 thubeamer.pdf 说明文档,文档查看方法:
%          下载模板文件夹里就有,如果是从 CTAN 上安装更新本模板,则通过
%          cmd 命令行:texdoc thubeamer 查看文档
%          推荐前往模板的 GitHub 仓库获取最新文件,地址:
%          https://github.com/YangLaTeX/thubeamer
%%=======================================================================

% 设置文档类别为 <beamer>
% \documentclass[aspectratio=169]{beamer} % 设置长宽比为 16:9
\documentclass{ctexbeamer}

% 使用 <thubeamer> 主题
% 模板选项如下
% (a.1) smoothbars: 页面顶端单行显示目录,默认选项
\usetheme[sectiontoc]{thubeamer}


% (a.2) sidebar: 页面左侧分栏显示目录
% \usetheme[sidebar]{thubeamer}

% (b) sectiontoc: 在每节(section)前显示目录,并高亮显示当前节,默认不显示

% (c) subsectiontoc: 在每小节(subsection)前显示目录,并高亮显示当前节和当前小节,默认不显示

% (d) en: 仅使用英文来制作 beamer,应用此选项后,汉字将全部无法编译

% 图片存放路径

\graphicspath{{figures/}}

\include{macros}

% 封面信息,方括号内容是显示在左侧边栏的内容(当选择 sidebar 主题时有效)
\title[报告标题]{篮球高光检测\\[2mm] Paper Reading}
\author[唐梓楠]{学生:唐梓楠\\[5mm] 导师:黄必清教授}
\institute[北京邮电大学]{\small  北京邮电大学, 清华大学}
\date{\small \vskip -10pt \today}

% 开始写文章
\begin{document}

% 标题页
\begin{frame}
  \maketitle
\end{frame}

% 目录页
\section*{目录}
\frame{
  \frametitle{\secname}
  \tableofcontents[hideallsubsections]
}

\section[DHMDL]{DHMDL: Dynamically Hashed Multimodal Deep Learning Framework for Racket Video Summarization Using Audio and Visual Markers}

\subsection{Motivation}

\begin{frame}{Motivation}
  \begin{block}{Applied Artificial Intelligence}
    \textbf{DHMDL: Dynamically Hashed Multimodal Deep Learning Framework for Racket Video Summarization Using Audio and Visual Markers}~\cite{priyankaDHMDLDynamicallyHashed2025}
  \end{block}

  \begin{itemize}
    \item 体育赛事视频在社交媒体平台广泛传播,用户对自动生成赛事精彩集锦需求强烈。但现有研究多聚焦\textbf{足球、板球}等运动,对\textbf{网球}等球拍类运动视频摘要关注不足。
    \item 人工剪辑赛事精彩瞬间耗时费力;现有视频摘要技术多依赖单一视觉模态,且缺乏高效数据结构与同步并行处理机制,难以处理长达数小时的赛事视频。
    \item 生成的赛事集锦可用于社交媒体短视频、精彩片段推送等场景,提升用户参与度与平台收益,还能促进体育联盟、社交媒体平台与内容创作者的合作。
  \end{itemize}
\end{frame}

\subsection{Methodology}

\begin{frame}{Methodology}
  框架分为\textbf{预处理}、\textbf{多模态特征提取与兴奋度评分}、\textbf{加权融合}、\textbf{动态哈希排序}、\textbf{亮点提取}五个模块:
  \begin{enumerate}
    \item 输入完整运动视频,预处理阶段将视频分割为 6 秒音频块和 1 帧 / 秒的视频帧。
    \item 并行提取\textbf{音频}(观众欢呼、解说员兴奋度)与\textbf{视觉}(运动员表情)标记的兴奋度评分。
    \item 加权融合多模态兴奋度评分,生成总兴奋度评分。
    \item \textbf{基于归并排序的动态哈希表}对总评分分类排序,快速定位高兴奋度亮点片段。
    \item 根据用户需求的摘要时长,提取并生成最终视频摘要。
  \end{enumerate}
\end{frame}

\begin{frame}{所提系统的详细系统结构图}
  \begin{figure}
    \includegraphics[width=0.9\linewidth]{DHMDL_pipeline.png}
  \end{figure}
\end{frame}

\begin{frame}{音频处理}
  \begin{itemize}
    \item \textbf{观众欢呼声}:\begin{itemize}
      \item 转换音频为\textbf{梅尔频谱图},提取\textbf{梅尔频率倒谱系数(MFCC)}\begin{equation}
        M(f) = 1125 \ln \left( 1 + \frac{f}{700} \right)
      \end{equation}
      \item \textbf{自定义 CNN 模型}(3 层卷积 + 池化 + 全连接层)分类 “欢呼 / 无欢呼”,输出 0-1 评分
    \end{itemize}
    \item \textbf{解说员兴奋度}:\begin{itemize}
      \item \textbf{音频处理}:分离前景(解说员声音)与背景音频,\textbf{自定义 CNN} 分类 “无声音 / 轻声 / 中等兴奋 / 高声兴奋”(4 类)
      \item \textbf{文本处理}:语音转文字提取解说文本,\textbf{LSTM 模型}识别 “兴奋词”(如 “精彩回击”“关键得分”),输出 0-1 评分
      \item 融合音频与文本评分,取平均值
    \end{itemize}
  \end{itemize}
\end{frame}

\begin{frame}{CNN 架构}
  \begin{figure}
    \includegraphics[width=\linewidth]{DHMDL_audio.png}
  \end{figure}
\end{frame}

\begin{frame}{视频处理}
    \textbf{运动员表情}:\begin{itemize}
      \item \textbf{多任务级联卷积网络(MTCNN)}检测含运动员面部的视频帧:\begin{enumerate}
        \item \textbf{P-Net(Proposal Network)}:快速生成面部候选框,输出含面部概率(0-1)、候选框坐标
        \item \textbf{R-Net(Refine Network)}:优化候选框 + 过滤误检,输出修正后的候选框、面部概率、5 个面部关键点(双眼、鼻尖、嘴角)
        \item \textbf{O-Net(Output Network)}:精准定位 + 输出最终结果,输出最终面部框、高置信度面部概率、5 个关键点坐标
      \end{enumerate}
      \item \textbf{提取特征}:手工特征(\textbf{HOG 梯度}方向捕捉形状与边缘、\textbf{LBP 纹理}捕捉局部纹理)+\textbf{3 种尺度 CNN 特征}($110 \times 110$、$90 \times 90$、$64 \times 64$ 输入)
      \item \textbf{SVM} 分类表情为 “开心 / 悲伤 / 中性 / 其他”,“开心”(高兴奋)与 “悲伤”(低兴奋)对应 0-1 评分
    \end{itemize}
\end{frame}

\begin{frame}{SVM 分类器架构}
  \begin{figure}
    \includegraphics[width=0.9\linewidth]{DHMDL_video.png}
  \end{figure}
\end{frame}

\begin{frame}{加权融合}
  \begin{equation}
    F(x)=\sum_{n=1}^{N} w_{n} E_{n}(x)
  \end{equation}
  其中 $w_{n}$ 为各模态权重(经自监督学习优化:观众欢呼 0.3、解说员语气 0.25、解说文本 0.15、运动员表情 0.3),$E_{n}(x)$ 为各模态兴奋度评分。
  
  \begin{block}{优化检索}
    利用\textbf{动态哈希表 + 归并排序}快速定位满足剪辑时长要求的高兴奋度片段。
  \end{block}
\end{frame}

\subsection{Datasets}

\begin{frame}{Datasets}
  \begin{itemize}
    \item \textbf{主数据集}:2019 年\textbf{美网和温网}完整比赛视频(时长 2 小时以上),分割为 6s 音频块(共 1619 个)和视频帧(共 43200 帧),80\% 用于训练,20\% 用于测试。
    \item \textbf{辅助数据集}:
      \begin{itemize}
        \item \textbf{音频分类}:标注 “欢呼 / 无欢呼” 样本 581 个、标注 “解说员兴奋度” 样本 1619 个。
        \item \textbf{表情识别}:CK + 面部表情数据集(593 张图像,8 类表情:中性、开心、悲伤等),用于训练运动员表情分类模型。
        \item \textbf{对比基准}:VSUMM、SumMe、TVSum 等通用视频摘要数据集,以及 YouTube 上的网球比赛人工标注亮点视频。
      \end{itemize}
  \end{itemize}
\end{frame}

\section[Event Camera]{Real-Time Ball Tracking and Action Classification using an Event Camera}

\subsection{Motivation}

\begin{frame}{Motivation}
  \begin{block}{ACM Multimedia 2025 MMSports Workshop}
    \textbf{Real-Time Ball Tracking and Action Classification using an Event Camera}~\cite{yamaneRealTimeBallTracking2025}
  \end{block}

  \begin{itemize}
    \item 体育赛事中回放片段对提升观众参与度至关重要,但传统回放筛选与编辑依赖人工,给现场工作人员带来沉重操作负担。
    \item 现有计算机视觉方法多缺乏实时性,无法满足自动回放生成需求,且多数方法针对特定运动、依赖辅助数据(如解说音频、过渡标识),实用性受限。
    \item 针对\textbf{排球赛事},提出一种基于事件相机的实时球类追踪与动作分类方法,解决自动回放生成的技术缺口,无需机器学习技术,利用事件数据流实现轻量化计算,最终识别适合回放的关键时刻。
  \end{itemize}
\end{frame}

\subsection{Methodology}

\begin{frame}{数据预处理}
  \begin{itemize}
    \item \textbf{数据格式转换}:\begin{itemize}
      \item \textbf{事件相机}输出 $(x, y, t, p)$ 形式数据(空间坐标、时间戳、极性),将 $\Delta t$ 时间内生成的事件聚合为单个事件帧;
      \item 若球预期区域的事件数低于阈值 $e_{\text{min}}$(实验设为 100),则合并下一帧事件,使帧时长变为 $2 \Delta t$,解决球减速(如接球后轨迹顶点)时事件少、难检测的问题。
    \end{itemize}
    \item \textbf{事件聚类生成}:\textit{通过形态学操作将离散的同物体事件分组,形成 “事件簇(EC)”},排除过小的事件簇(视为相机噪声)。
  \end{itemize}
  \begin{block}{事件相机优点}
    异步触发式视觉传感器,仅在像素级检测到 “亮度变化” 时才生成数据,而非周期性采集完整图像,这种特性使其在高速运动、高动态范围场景中具备传统相机难以比拟的优势。
  \end{block}
\end{frame}

\begin{frame}{初始位置确定}
  监测预定义发球区,若连续 3 帧检测到重叠事件簇,其中心即为球的初始位置
  \begin{figure}
    \includegraphics[width=0.8\linewidth]{ball_start.png}
    \caption{左侧和右侧预先定义的发球区(以绿色标注,本示例中发球从左侧开始)。}
  \end{figure}
\end{frame}

\begin{frame}{实时追踪逻辑}
  \begin{enumerate}
    \item 预设球半径 $r$(实验设为 10),以前一帧事件簇中心 $(c_x, c_y)$ 定义球区域,当前帧与该区域空间相交的事件簇视为球(不使用 IoU,因球与球员接触时事件簇易融合,IoU 得分低)。
    \item \textbf{事件簇融合处理}:若球与球员事件簇融合,通过事件密度提取球区域(球区域事件密度更高);若球沿抛物线运动(接球、传球时),事件数低于 $e_{\text{min}}$ 则合并下一帧事件。
    \item \textbf{未检测应对}:连续 3 帧未检测时,以前一帧球位置为中心,用预设大小窗口 $w$(实验设为 20,即 $2r$)搜索最高密度事件簇;连续 10 帧未检测则扩展至全图搜索,通过密度上限 $d_{\text{max}}$(3.0)和面积下限 $S_{\text{min}}$(200)避免误检(如观众区域、背景噪声)。
  \end{enumerate}
\end{frame}

\begin{frame}
  \begin{figure}
    \includegraphics[width=\linewidth]{ball_3.png}
    \caption{扣球动作中球与球员的事件簇出现融合的场景示例。绿色方框代表事件簇发生融合的区域。(b)图显示,即便事件簇发生融合,球所在区域的事件数量仍高于其他区域。}
  \end{figure}
\end{frame}

\subsection{Results}

\begin{frame}{判断方法}
  无需机器学习模型,通过轻量化的轨迹分析与规则判断实现。
  \begin{itemize}
    \item 识别动作起始(轨迹突变触发)
    \item 基于轨迹特征区分动作类 \begin{itemize}
      \item 优先判定 “扣球”:$x$ 向位移 > $y$ 向位移
      \item 区分 “传球” 与 “接球”:依赖后续动作
      \item 直接判定 “发球”:回合起始 + 高轨迹
    \end{itemize}
  \end{itemize}
\end{frame}

\begin{frame}{Results}
  \textbf{Datasets}:自主采集工业排球联赛事件数据(Prophesee Gen4 相机,$1280 \times 720$ 分辨率),同步采集 120fps RGB 数据(GoPro)用于对比;数据集片段从发球开始到球落地得分,人工标注动作真值。

  \begin{table}[h]
    \centering
    \resizebox{\textwidth}{!}{%
    \begin{tabular}{lccc}
      \toprule
      action & accuracy & transition error (s) & analysis \\
      \midrule
      发球 & 0.99 & - & 回合起始、轨迹高,无遮挡 \\
      接球 & 0.67 & 0.033 & 多在地面附近,球与球员重叠多,检测难 \\
      传球 & 0.82 & 0.022 & 高空执行,遮挡少,但速度慢易与球员混淆 \\
      扣球 & 0.97 & 0.178 & 速度快、轨迹特征明显,过渡误差高因球员与球接触时间长 \\
      \bottomrule
    \end{tabular}%
    }
  \end{table}
\end{frame}

\section[Player Tracking and Re-Id]{Enhancing Soccer Player Tracking and Re-Identification with Dual Visual Embeddings and Tracklet Association}

\subsection{Pipeline}

\begin{frame}{Pipeline}
  \begin{columns}
    \column{0.5\textwidth}
    \begin{block}{ACM Multimedia 2025 MMSports Workshop}
      \textbf{Enhancing Soccer Player Tracking and Re-Identification with Dual Visual Embeddings and Tracklet Association}~\cite{nakamuraEnhancingSoccerPlayer2025}
    \end{block}
    输入视频 $\to$ RF-DETR 检测 $\to$ CLIP-ReID+PRT-ReID 特征提取 $\to$ BoT-SORT 局部关联 $\to$ GTA 全局轨迹合并 $\to$ 插值 + 平滑后处理 $\to$ 输出最终球员轨迹
    \column{0.5\textwidth}
    \begin{figure}
      \includegraphics[height=0.8\textheight]{pipeline.png}
    \end{figure}
  \end{columns}
\end{frame}

\subsection{Methodology}

\begin{frame}{目标检测:RF-DETR}
    \begin{itemize}
      \item \textbf{优势}:基于 Transformer 架构,配备可变形注意力与 DINOv2 骨干网络,捕捉全局上下文信息,优于依赖局部区域特征的传统 CNN 检测器(YOLO、Faster R-CNN)。
      \item \textbf{训练细节}:用 COCO 预训练权重初始化,在 SoccerTrack 训练集 1/10 随机采样帧上微调;为平衡泛化性与针对性,仅在小规模随机子集上微调,避免过拟合某一场地场景。
      \item \textbf{作用}:高精度检测小尺度球员,即使球员仅占数十像素仍保持高召回率,通过置信阈值抑制假阳性。
    \end{itemize}
\end{frame}

\begin{frame}{多维度特征提取:CLIP-ReID + PRT-ReID}
    \begin{itemize}
      \item \textbf{CLIP-ReID}:直接采用 CLIP 视觉编码器,生成语义丰富、语言关联的外观嵌入,抗光照变化与姿态变异。
      \item \textbf{PRT-ReID}:基于部件的多任务模型,联合学习球员角色与球队属性,输出细粒度部件级与全局外观特征。
      \item \textbf{设计初衷}:融合 CLIP 的广谱语义理解与 PRT-ReID 的细粒度区分能力,提升后续关联阶段的身份一致性。
    \end{itemize}
\end{frame}

\begin{frame}{局部帧间关联:BoT-SORT}
    \begin{itemize}
      \item \textbf{改进点}:在经典 SORT(仅运动预测)、DeepSORT(运动预测 + CNN-ReID)基础上,集成卡尔曼滤波运动预测、CLIP-ReID 嵌入与相机运动补偿(CMC)。
      \item \textbf{作用}:校正动态相机导致的帧间位置偏移,实现实时、稳健的帧间轨迹链接,抵抗短期遮挡与虚假检测,减少身份切换。
    \end{itemize}
\end{frame}

\begin{frame}{全局轨迹段合并:GTA(Global Tracklet Association)}
    \begin{itemize}
      \item \textbf{解决问题}:弥补 BoT-SORT 在长期遮挡或检测缺失下的轨迹碎片化问题。
      \item \textbf{流程}:先通过类 DBSCAN 方法聚类轨迹段,再结合 PRT-ReID 外观距离(部件级线索)与检测框空间距离,加权合并聚类结果。
      \item \textbf{效果}:在挑战数据集上使轨迹段数量减少约一半,提升跟踪性能指标。
    \end{itemize}
\end{frame}

\begin{frame}{后处理:补全与平滑}
    \begin{itemize}
      \item 异常值处理:计算轨迹段瞬时速度,标记偏离近期平均值的异常值。
      \item 插值补全:对容忍范围内的轨迹间隙,线性插值补全缺失的 bounding box;过滤过短轨迹段。
      \item 平滑优化:对每个轨迹段应用移动平均滤波,消除位置与尺寸的突变,生成贴合球员真实运动的流畅轨迹。
    \end{itemize}
\end{frame}

\subsection{Results}

\begin{frame}{Results}
  \begin{table}[htbp]
    \centering
    \resizebox{\textwidth}{!}{%
    \begin{tabular}{lcc}
      \toprule
      指标 & 数值 & 指标含义 \\
      \midrule
      HOTA & 0.55 & 综合检测与关联精度的核心指标 \\
      LocA & 0.82 & 定位精度,反映 RF-DETR 对小尺度球员的稳定检测能力 \\
      AssA & 0.43 & 关联精度,体现 BoT-SORT 与 GTA 在遮挡后维持身份一致性的效果 \\
      \bottomrule
    \end{tabular}%
    }
  \end{table}
  \begin{table}[htbp]
    \centering
    \resizebox{\textwidth}{!}{%
    \begin{tabular}{lcccccccc}
      \toprule
      Pipeline & Detection & Feature Extraction & Tracking & GTA & Post Processing & HOTA & IDSW & IDF1 \\
      \midrule
      Proposed (Fusion+GTA+Interp) & RF-DETR & PRT-ReID+CLIP-ReID & BoT-SORT & \checkmark & \checkmark & 51.486 & 446 & 59.041 \\
      \midrule
      RF-DETR@560 (Fusion+GTA+Interp) & RF-DETR & PRT-ReID+CLIP-ReID & BoT-SORT & \checkmark & \checkmark & 28.867 & 1524 & 33.641 \\
      CLIP-Only (CLIP+GTA+Interp) & RF-DETR & CLIP-ReID & BoT-SORT & \checkmark & \checkmark & 52.534 & 414 & 60.925 \\
      PRT-Only (PRT+GTA+Interp) & RF-DETR & PRT-ReID & BoT-SORT & \checkmark & \checkmark & 48.525 & 467 & 56.240 \\
      No-GTA (Fusion+Interp) & RF-DETR & PRT-ReID+CLIP-ReID & BoT-SORT & & \checkmark & 50.581 & 294 & 55.884 \\
      No-Interp (Fusion+GTA) & RF-DETR & PRT-ReID+CLIP-ReID & BoT-SORT & \checkmark & & 51.100 & 729 & 58.750 \\
      \bottomrule
    \end{tabular}%
    }
  \end{table}
\end{frame}

\begin{frame}[allowframebreaks]{参考文献}
  \bibliographystyle{thubeamer}
  \bibliography{reference}
\end{frame}

\begin{frame}
  \begin{center}
    {\Huge\calligra Thanks for your attention!}
  \end{center}
\end{frame}

% 结束文档撰写
\end{document}
